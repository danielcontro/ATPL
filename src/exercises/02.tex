\subsection{}

Think about the difference between the terms \FN{x}{\SUM{2}{x}} and
\FN{x}{\APP{(\FN{y}{(\SUM{y}{x})})}{2}}.

Mathematically they are the same function. The first one is exactly
the second one after one derivation step, according to the (BETA)
rule.

Write a term rapresenting an higher order function, i.e. a function
that takes a function as a parameter. Also write a program that
somehow \textit{uses} a higher order function. And finally write a
function that returns a function.

\begin{itemize}
\item Higher order function: \(\FN{x}{\FN{y} (\APP{x}{y})}\) (call it
  \(F\))
\item function that uses an higher order function
  \(\FN{z}{\APP{(\APP{(F)}{(\FN{w}{\SUM{w}{2}})})}{(z)}}\)
\item function that returns a function \(\FN{x}{\FN{y}{y}}\)
\end{itemize}

Find 2 terms \(M_1, M_2\) corresponding to scala \texttt{sum} and
\texttt{sumThree}

\begin{itemize}
\item[\textbf{sum}]      \FN{x}{\FN{y}{(x + y)}}
\item[\textbf{sumThree}] \APP{sum}{(3)}
\end{itemize}

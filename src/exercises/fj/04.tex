\subsection{}

Why are there two typing rules for both upcast and downcast, while there's only one rule for
upcast in the operational semantics?\\~\\
Int the typing rules it is necessary to check whether the types to which terms are casted are in
the same chain of the hierarchy in order to check if the operations make sense.
If it was allowed downcasting in the operational semantics of an object to a subtype of its
class, the evaluation could evolve to a stuck term in the case that methods or fields of the
subclass, not present in the superclass, were referenced thus violating the safety theorem,
upcasting instead can't lead to such situations hence it's allowed. 

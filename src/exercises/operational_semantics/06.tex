\subsection{}

Examine how the semantics of the language would change by replacing the
(SUM-RIGHT) rule with the following one:
\[
	\infer[\text{\small{(SUM-RIGHT-NEW)}}]
	{\SUM{M_1}{M_2} \to \SUM{M_1}{M_2'}}
	{M_2 \to M_2'}
\]
Is this new modified semantics deterministic? Is it possible for a term $M$ to
evolve to two different (final) values? If a program reduces to a value or a
stuck term, respectively, in the original semantics, does it now reduce to
the same value or stuck term?\\~\\
The modified semantics is no longer deterministic, for example the term
$\SUM{(\SUM{3}{4})}{(\SUM{2}{5})}$ can be reduced by applying either the
(SUM-LEFT) or the (SUM-RIGHT') rule.\\
If a term reduces to a value in the original semantics then it will reduce the
same one in the updated one since the number of (SUM-LEFT) and (SUM-RIGHT-NEW)
will be the same of the original semantics.
Instead if a term reduces to a stuck term in the original semantics it could
reduce to a different stuck term in the updated semantics, because if the
stuck subterm is on the left side of the sum the right side could be partially
or totally reduced according to the (SUM-RIGHT-NEW) rule, thus leading to a
different stuck term once the left side will be reduced.

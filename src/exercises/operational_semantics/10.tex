\subsection{}

Consider the following function definitions in Scala:
\begin{itemize}
	\item \texttt{def square(x:Int):Int = x*x}
	\item \texttt{def sumOfSquare(x:Int, y:Int):Int = square(x)+square(y)}
\end{itemize}
\begin{enumerate}
	\item Describe the reduction steps of the expression \texttt{sumOfSquare(3,4)}
	      according to a call-by-value strategy similar to the one defined in the
	      previous section. Also describe the reduction of the same expression
	      according to the call-by-name strategy.
	\item Describe the reduction steps of the expression
	      \texttt{sumOfSquare(3,2+2)} according to the call-by-value and
	      call-by-name strategies.
\end{enumerate}
\begin{enumerate}
	\item call-by-value:
	      \begin{align*}
		       & \texttt{sumOfSquare(3,4)} \to \texttt{square(3)+square(4)} \to \\
		       & \texttt{(3*3)+square(4)} \to \texttt{9 + square(4)} \to        \\
		       & \texttt{9 + (4 * 4)} \to \texttt{9 + 16} \to \texttt{25}
	      \end{align*}
	      call-by-name
	      \begin{align*}
		       & \texttt{sumOfSquare(3,4)} \to \texttt{square(3)+square(4)} \to \\
		       & \texttt{(3*3)+square(4)} \to \texttt{9 + square(4)} \to        \\
		       & \texttt{9 + (4 * 4)} \to \texttt{9 + 16} \to \texttt{25}
	      \end{align*}
	\item call-by-value:
	      \begin{align*}
		      \texttt{sumOfSquare(3,2+2)} \to \texttt{sumOfSquare(3, 4)} \to ...
	      \end{align*}
	      call-by-name:
	      \begin{align*}
		       & \texttt{sumOfSquare(3,2+2)} \to \texttt{square(3)+square(2+2)} \to                \\
		       & \texttt{(3*3)+square(2+2)} \to \texttt{9 + ((2+2)*(2+2))} \to                     \\
		       & \texttt{9 + (4*(2+2))} \to \texttt{9 + (4*4)} \to \texttt{9 + 16} \to \texttt{25}
	      \end{align*}
\end{enumerate}

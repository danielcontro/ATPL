\subsection{}

Consider the logical construct \texttt{x\&\&y} commonly used in the programming
languages, which only evaluates the second argument if $x$ is true:
\begin{itemize}
\item define a term $M_\texttt{\&}$ in the language $\mathcal{L}$ that
  behaves as $x\texttt{\&\&y}$. Does the term $M_\texttt{\&}$ respect
  the intended semantics of the construct $x\texttt{\&\&}y$ only with
  the call-by-value evaluation strategy, only with the call-by-name
  strategy, with both?\\~\\
  $M_\texttt{\&} = \FN{x}{(\FN{y}{\IF{x}{y}{false}})}$\\
  The statement respects  the intended semantics only in a call-by-name
  strategy since with a call-by-value evaluation strategy $y$ will be
  evaluated before applying the (BETA) rule to the term
  $\FN{y}{\IF{x}{y}{false}}$ regardless of the value of $x$.
\item Define in Scala a function and that behaves as the logical construct
  $x\texttt{\&\&}y$
  \begin{align*}
    & \texttt{def and(x: => Bool, y: => Bool): Bool =} \\
    & \qquad\texttt{if }x\texttt{ then }y              \\
    & \qquad\texttt{else }false
  \end{align*}
\end{itemize}

\subsection{}

If we had defined the subtyping rules in such a way that $\TREC{l: Nat} <: \TREC{l: Nat, l': Nat}$
which theorem would no longer be true? Identify it and give a counterexample. If we did not have
the rule (ARROW), how would the type system change?\\~\\
If the subtyping relation would hold in a different typing system the Progress Theorem wouldn't
hold anymore, let $M = \SEL{\REC{l=1}}{l'}$, thus
\[
	\infer[\text{\small{(TYPE SELECT)}}]
	{\emptyset \vdash \SEL{\REC{l=1}}{l'}: Nat}
	{
		\infer[\text{\small{(SUBSUMPTION)}}]
		{\emptyset \vdash \REC{l=1}: \TREC{l: Nat,l': Nat}}
		{
			\infer[\text{\small{(TYPE RECORD)}}]
			{\emptyset \vdash \REC{l=1}: \TREC{l: Nat}}
			{
				\infer{\emptyset \vdash 1: Nat}{\text{\small{(T-INT)}}}
			} &
			\infer{ \TREC{l: Nat} <: \TREC{l: Nat, l': Nat} }{\vdots}
		}
	}
\]
But $M$ is neither a value, nor it exists $M'$ such that $M \to M'$.\\
Removing a typing rule such as (ARROW) reduces the set of well-typed
programs hence the properties of the type system holds still but for less terms, namely functions
that returns a supertype of the return type won't be well-typed as in the original typing system
with subtyping.

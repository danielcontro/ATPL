\subsection{}

Prove the theorem of type preservation, progress and safety for the
base language extended with records and subtyping.

\paragraph*{Progress theorem:\\}

If \(\emptyset \vdash M: T\) then either \(M\) is a value or \(\exists\,
M'.\, M \to M'\).\\ By induction on the height of the derivation of
\(\emptyset \vdash M: T\): The base cases are left unchanged since no
new typing axiom has been added, only the inductive ones has to be
updated, based on the last rule of the derivation applied.
\begin{itemize}
\item (TYPE RECORD): then \(\emptyset \vdash \REC{l_i=M_i^{i \in
    1..n}}: T\), by Inv. Lemma \(\exists\, S_1..S_n.\,\forall\, i \in
  1..n\ \emptyset \vdash M_i: S_i\) and \(\TREC{l_i: S_i^{i \in 1..n}}
  <: T\) with height of the derivation \(\le k_0\) thus the inductive
  hypothesis can be applied, obtaining that \(\forall\, i \in 1..n M_i\)
  is either a value or \(\exists\, M_i'.\, M_i \to M_i'\). If all \(M_i\)
  are values \(v_i\) then \(\REC{l_i=v_i^{i \in 1..n}}\) is a value,
  otherwise if \(\exists\, k \in 1..n.\exists\, M_k'.\, M_k \to M_k'\)
  then the (EVAL RECORD) rule can be applied, hence \(\exists\, M' =
  \REC{l_j=M_j^{j \in 1..k-1}, l_k = M_k', l_i = M_i^{i \in
      k+1..n}}.\, M \to M'\)
\item (TYPE SELECT): then \(\emptyset \vdash \SEL{N}{l_j}: T\), by
  Inv. Lemma \(\exists\, n.\, j \in 1..n\) and \(\forall\, i \in 1..n.\,
  \exists\, l_i, T_i.\, \emptyset \vdash N: \TREC{l_i: T_i^{i \in
      1..n}}\) and \(T = T_j\) with height of the derivation \(\le k_0\)
  hence the inductive hyp. can be applied obtaining that either \(N\) is
  a value, in which case the (SELECT) rule can be applied, or
  \(\exists\, N'.\, N \to N'\), hence the (EVAL SELECT) rule applies.
\item (SUBSUMPTION): then \(\emptyset \vdash M: T\), by "Inv. Lemma"
  \(\emptyset \vdash M: S\) and \(S <: T\) with height of the derivation
  \(\le k_0\) hence the inductive hyp. can be applied obtaining that \(M\)
  is either a value or \(\exists\, M'.\, M \to M'\)\qed
\end{itemize}

\paragraph*{Substitution Lemma:\\}
If \(\Gamma, x: S \vdash M: T\) and \(\Gamma \vdash N: S\) then \(\Gamma
\vdash M\SUBST{x}{N}: T\).  The base cases are left unchanged, only the
inductive ones has to be extended with the newly added typing rules.
By induction on the height of the derivation of \(\Gamma, x: S \vdash
M: T\):
\begin{itemize}
\item (TYPE RECORD): then \(\Gamma, x: S \vdash \REC{l_i=M_i^{i \in
    1..n}}: T\), by Inv. Lemma \(\exists\, S_1..S_n.\,\forall\, i \in
  1..n\ \Gamma, x: S \vdash M_i: S_i\) and \(\TREC{l_i: S_i^{i \in
      1..n}} <: T\) with height of the derivation \(\le k_0\) thus the
  inductive hypothesis can be applied, obtaining that \(\forall\, i \in
  1..n \Gamma \vdash M_i\SUBST{x}{N}: S_i\), hence the (TYPE RECORD)
  rule can be applied getting \(\Gamma \vdash \REC{l_i =
    M_i\SUBST{x}{N}^{i \in 1..n}} = M\SUBST{x}{N} : T\)
\item (TYPE SELECT): then \(\Gamma, x: S \vdash \SEL{M_1}{l_j}: T\), by
  Inv. Lemma \(\exists\, n.\, j \in 1..n\) and \(\forall\, i \in 1..n.\,
  \exists\, l_i, T_i.\, \Gamma, x: S \vdash M_1: \TREC{l_i: T_i^{i \in
      1..n}}\) and \(T = T_j\) with height of the derivation \(\le k_0\)
  hence the inductive hyp. can be applied obtaining that \(\Gamma
  \vdash M_1\SUBST{x}{N}: \TREC{l_i: T_i^{i \in 1..n}}\) hence the
  (TYPE SELECT) rule can be applied obtaining \(\Gamma \vdash
  \SEL{M_1\SUBST{x}{N}}{l_j} = M\SUBST{x}{N}: T_j\)
\item (SUBSUMPTION): then \(\Gamma, x:S \vdash M: T\), by "Inv. Lemma"
  \(\Gamma, x: S \vdash M: U\) and \(U <: T\) with height of the
  derivation \(\le k_0\) hence the inductive hyp. can be applied
  obtaining \(\Gamma \vdash M\SUBST{x}{N}: U\) hence the (SUBSUMPTION)
  rule can be applied getting \(\Gamma \vdash M\SUBST{x}{N}: T\)\qed
\end{itemize}

\paragraph*{Type preservation:\\}
If \(\Gamma \vdash M: T\) and \(M \to M'\) then \(\Gamma \vdash M': T\).
The base cases (SUM), (IF-TRUE) and (BETA) are left unchanged, we have
to add a new base case:
\begin{itemize}
\item (SELECT) was the lastly applied rule, it menas that the
  conclusion is \(\REC{\ell_i = v_i \quad^{i\in 1\dots n}}.\ell_j \to
  v_j\) by hypothesis we know \[\Gamma \vdash \REC{\ell_i = v_i
    \quad^{i\in 1\dots n}} \TYPE{\REC{\ell_i \TYPE{T_i} \quad^{i\in
        1\dots j-1}, \ell_j \TYPE{T_j}, \ell_k \TYPE{T_k} \quad^{k\in
        j+1\dots n}}}\] so by inversion lemma we have \(\forall i \in
  1 \dots n \quad \Gamma\vdash v_i\TYPE{T_i}\) which in particular
  means \(j\in 1\dots n \Rightarrow \Gamma \vdash v_j \TYPE{T_j}\)
  which is our thesis
\end{itemize}

Inductive cases, working by induction on the height of the derivation
of \(M \to M'\) also have to be updated with the new evaluation rules:
\begin{itemize}
\item (EVAL SELECT) was the lastly applied rule, therefore the
  conclusion is \[M.\ell_j \to M'.\ell_j\] and the premise is \(M\to
  M'\) which has an height of derivation at most \(k\). By considering
  the hypothesis \(\Gamma\vdash M.\ell_j \TYPE{T_j}\) we have by
  inversion lemma that \(\Gamma\vdash M \TYPE{T} \mid T <: \REC{\ell_j
    : T_j}\). We can apply the inductive hypothesis and say that
  \(\Gamma\vdash M' \TYPE{T} \mid T <: \REC{\ell_j \TYPE{T_j}}\) but
  this means by (TYPE SELECT) that \(\Gamma \vdash M'.\ell_j
  \TYPE{T_j}\) which is our thesis.
\item (EVAL RECORD) was the lastly applied rule, therefore the
  conclusion (\(M \to M'\)) is \[\REC{\ell_i = v_i \quad^{i\in 1\dots
      j-1}, \ell_j = M_j, \ell_k = M_k \quad^{k\in j+1\dots n}} \to
  \REC{ \ell_i = v_i \quad^{i\in 1\dots j-1}, \ell_j = M_j', \ell_k =
    M_k \quad^{k\in j+1\dots n}}\] and the hypothesis is that \(M_j\to
  M_j'\), derivable with a derivation tree of height at most \(k\). by
  hypothesis \(\Gamma \vdash M \TYPE{T}\), which by inversion lemma
  means \(T = \REC{\ell_i \TYPE{T_i} \quad^{i \in 1 \dots n}}\), which
  is derivable, so by inversion lemma again we have that \(\forall i
  \in 1 \dots n \quad \Gamma\vdash M_i \TYPE{T_i}\), in particular
  \(\Gamma \vdash M_j \TYPE{T_j}\). By induction hypothesis \(\Gamma
  \vdash M_j' \TYPE{T_j}\), which means \[\Gamma \vdash \REC{\ell_i =
    v_i \quad^{i\in 1 \dots j-1}, \ell_j = M_j', \ell_k = M_k
    \quad^{k\in j+1, \dots n}} \TYPE{\REC{\ell_i \TYPE{T_i}
      \quad^{i\in 1 \dots j-1}, \ell_j \TYPE{T_j}, \ell_k \TYPE{T_k}
      \quad^{k\in j+1, \dots n}}}\] which is exactly \(T\), which
  makes the thesis.
\end{itemize}

\subsection{}

Prove that if $\Gamma \vdash M: T$ is a derivable judgement, then
$\fv{M} \subseteq Dom(\Gamma)$.\\~\\
The proof will proceed by induction on the height of the derivation tree of the judgement
$\Gamma \vdash M: T$.

\paragraph*{Base cases: $h = 1$:\\}

The base cases are those where the last typing rule is an axiom:

\begin{itemize}
	\item (T-TRUE): if $\Gamma \vdash true: Bool$ is derivable then\\
	      $\fv{true} = \emptyset \subseteq Dom(\Gamma)$ holds
	\item (T-FALSE): if $\Gamma \vdash false: Bool$ is derivable then\\
	      $\fv{true} = \emptyset \subseteq Dom(\Gamma)$ holds
	\item (T-INT): if $\Gamma \vdash n: Nat$ is derivable then\\
	      $\fv{n} = \emptyset \subseteq Dom(\Gamma)$ holds
	\item (T-VAR): if $\Gamma \vdash x: T$ is derivable then\\
	      by side condition $x : T \in \Gamma$ thus $x \in Dom(\Gamma)$, hence
	      $\fv{x} = \{x \} \subseteq Dom(\Gamma)$ holds
\end{itemize}

\paragraph*{Inductive cases: $h = h_0 + 1$:\\}

The inductive hypothesis is that for any judgement $\Gamma \vdash M: T$ of height $h \le h_0$
$\fv{M} \subseteq Dom(\Gamma)$. The inductive cases will be distinguished based on the last
typing rule applied to the derivation tree:

\begin{itemize}
	\item (T-SUM): $M = \SUM{M_1}{M_2}$,
	      if $\Gamma \vdash M: T$
	      by Inversion Lemma:
	      \begin{itemize}
		      \item $\Gamma \vdash M_1 : Nat$ derivable
		      \item $\Gamma \vdash M_2 : Nat$ derivable
	      \end{itemize}
	      $\fv{M}=\fv{M_1}\cup\fv{M_2}$, since $M_1, M_2$ have a derivation tree of height $h_0$
	      the inductive hypothesis can be applied, and since subsets are closed under union
	      $\fv{M}=\fv{M_1}\cup\fv{M_2} \subseteq Dom(\Gamma)$ holds
	\item (T-IFTHENELSE): $M = \IF{M_1}{M_2}{M_3}$
	      if $\Gamma \vdash M: T$
	      by Inversion Lemma:
	      \begin{itemize}
		      \item $\Gamma \vdash M_1 : Bool$ derivable
		      \item $\Gamma \vdash M_2 : T$ derivable
		      \item $\Gamma \vdash M_3 : T$ derivable
	      \end{itemize}
	      $\fv{M}=\fv{M_1}\cup\fv{M_2}\cup\fv{M_3}$, since $M_1, M_2, M_3$ have a derivation tree of height $h_0$
	      the inductive hypothesis can be applied, and since subsets are closed under union
	      $\fv{M}=\fv{M_1}\cup\fv{M_2}\cup\fv{M_3} \subseteq Dom(\Gamma)$ holds
	\item (T-FUN): $M = \FN{x: T_1}{M_1}$
	      if $\Gamma \vdash M: T$
	      by Inversion Lemma:
	      \begin{itemize}
		      \item $\exists\, T_2.\, T = T_1 \to T_2$
		      \item $\Gamma, x: T_1 \vdash M_1 : T_2$ derivable
	      \end{itemize}
	      $\fv{M}=\fv{M_1}\setminus \{x\}$, since $M_1$ have a derivation tree of height $h_0$
	      the inductive hypothesis can be applied, thus $\fv{M_1}\subseteq Dom(\Gamma,x:T_1)$.
	      Moreover $\fv{M}=\fv{M_1}\setminus\{x\}$ hence $\fv{M}\subseteq Dom(\Gamma)$ holds
	\item (T-APP): $M = \APP{M_1}{M_2}$
	      if $\Gamma \vdash M: T$
	      by Inversion Lemma:
	      \begin{itemize}
		      \item $\Gamma \vdash M_1 : Nat$ derivable
		      \item $\Gamma \vdash M_2 : Nat$ derivable
	      \end{itemize}
	      $\fv{M}=\fv{M_1}\cup\fv{M_2}$, since $M_1, M_2$ have a derivation tree of height $h_0$
	      the inductive hypothesis can be applied, and since subsets are closed under union
	      $\fv{M}=\fv{M_1}\cup\fv{M_2} \subseteq Dom(\Gamma)$ holds
\end{itemize}



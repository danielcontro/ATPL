\subsection{}

If instead of the rule (T-APP) the following rule was defined, would the safety
theorem still be true?
\[
	\infer[\text{\small{(APP')}}]
	{\Gamma \vdash \APP{M}{N}: T}
	{
		\Gamma \vdash M: T \to T &
		\Gamma \vdash N: T
	}
\]

The Safety Theorem states that if $\emptyset \vdash M: T$ and $M \to^\ast M'.\, M' \not\to$ then
$M'$  is a value.
In order to prove it we have to prove the Progress Theorem and the the corollary of the
Subject Reduction Theorem with the updated typing rules.

\subsubsection*{Progress Theorem:}

If $\emptyset \vdash M: T$ then $M$ is a value or $\exists\, M'.\, M \to M'$
By induction on the height of the derivation tree of $\emptyset \vdash M: T$:

\paragraph*{Base cases:\\}
Analogous to the original proof

\paragraph*{Inductive cases $h = h_0 + 1$:\\}

The inductive hypothesis is that for all the derivations of height $h \le h_0$ if
$\emptyset \vdash M: T$ then $M$ is a value or $\exists\, M'.\, M \to M'$. The only case which
differs is if the last rule in the derivation is (T-APP') with $M = \APP{M_1}{M_2}$ thus the
conclusion of the rule is $\emptyset \vdash \APP{M_1}{M_2}: T$, and the premises are
$\emptyset \vdash M_1 : T \to T$ and $\emptyset \vdash M_2: T$ both of height at most $h_0$.
The inductive hypothesis can be applied to the first judgement thus:
\begin{itemize}
	\item $M_1$ value: by the lemma of Canonical Forms $M_1 = \FN{x:T}{M_{1_1}}$ hence
	      $M = \APP{\FN{x:T}{M_{1_1}}}{M_2}$. By applying the inductive hypothesis to the second
	      judgement either $M_2$ is a value $v_2$ hence the (BETA) rule can be applied obtaining
	      $M \to M_{1_1}\SUBST{x}{v_2}$ or (APP2) applies obtaining
	      $M \to \APP{\FN{x:T}{M_{1_1}}}{M_2'}$
	\item $M_1 \to M1'$: (APP1) applies, thus $M \to \APP{M_1'}{M_2}$
\end{itemize}

\subsubsection*{Substitution Lemma:}

If $\Gamma, x: S \vdash M: T$ and $\Gamma \vdash N: S$ then $\Gamma \vdash M\SUBST{x}{N}: T$
By induction on the height of the derivation of $\Gamma, x: S \vdash M: T$:

\paragraph*{Base cases:\\}
Analogous to the original proof

\paragraph*{Inductive cases $h = h_0 + 1$:\\}

The inductive hypothesis is that for all the derivations of height $h \le h_0$ if
$\Gamma, x: S \vdash M: T$ and $\Gamma \vdash N: S$ then $\Gamma \vdash M\SUBST{x}{N}: T$
is derivable.
The only case which differs is if the last rule in the derivation is (T-APP') with
$M = \APP{M_1}{M_2}$ thus the conclusion of the rule is $\Gamma, x: S \vdash \APP{M_1}{M_2}: T$,
and the premises are:
\begin{itemize}
	\item $\Gamma, x: S \vdash M_1 : T \to T$
	\item $\Gamma, x: S \vdash M_2: T$
\end{itemize}

The inductive hypothesis can be applied to both the premises obtaining
\begin{itemize}
	\item $\Gamma \vdash M_1\SUBST{x}{N}: T \to T$
	\item $\Gamma \vdash M_2\SUBST{x}{N}: T$
\end{itemize}
hence the (T-APP') rule can be applied obtaining
$\Gamma \vdash \APP{M_1\SUBST{x}{N}}{M_2\SUBST{x}{N}}: T$\qed

\subsubsection*{Subject Reduction Theorem:}

If $\Gamma \vdash M: T$ and $M \to M'$ then $\Gamma \vdash M': T$.
By induction on the height of the derivation of $\Gamma \vdash M: T$:

\paragraph*{Base cases:\\}

Analogous to the original proof

\paragraph*{Inductive cases $h = h_0 + 1$:\\}
The inductive hypothesis is that for all the derivations of height $h \le h_0$ if
$\Gamma \vdash M: T$ and $M \to M'$ then $\Gamma \vdash M':T$ is derivable. The only case which
differs is if the last rule in the derivation is (T-APP') with $M = \APP{M_1}{M_2}$ thus the
conclusion of the rule is $\Gamma \vdash \APP{M_1}{M_2}: T$, and the premises are
\begin{itemize}
	\item $\Gamma \vdash M_1 : T \to T$
	\item $\Gamma \vdash M_2: T$
\end{itemize}
if $M \to M'$ to get $M'$ it was applied one of the following operational semantics rules:
\begin{itemize}
	\item (BETA): $M = \APP{(\FN{x: T}{M_{1_1}})}{v} \to M\SUBST{x}{v}$, by Inversion
	      Lemma'\\
	      $\Gamma \vdash \FN{x:T}{M_{1_1}}: T \to T$ and $\Gamma \vdash v: T$.
	      By Inversion Lemma' once again on $M_1$ $\Gamma, x: T \vdash M_{1_1}: T$.
	      Now by Substitution Lemma' $\Gamma \vdash M\SUBST{x}{v}\;(= M'): T$
	\item (APP1): $M \to \APP{M_1'}{M_2} = M'$ since $\Gamma \vdash M_1: T \to T$
	      derivable by Inv. Lemma' with height $h_0$ the inductive hypothesis can be applied,
	      hence $\Gamma \vdash M_1': T \to T$ derivable. Thus by (T-APP') rule
	      $\Gamma \vdash M': T$ derivable
	\item (APP2) $M \to \APP{v}{M_2'} = M'$ since $\Gamma \vdash M_2: T$ derivable by
	      Inv. Lemma with height $h_0$ the inductive hypothesis can be applied,
	      hence $\Gamma \vdash M_2': T$ derivable. Thus by (T-APP') rule
	      $\Gamma \vdash M': T$ derivable
\end{itemize}

\subsubsection*{Corollary of Subject Reduction Theorem:}

If $\emptyset \vdash M: T$ and $M \to M'$ then $\emptyset \vdash M': T$
By induction on the number of the reduction steps $k$ in $M \to M'$.

\paragraph*{$k = 0$:} $M' = M$ thus immediate
\paragraph*{$k = k_0 + 1$:\\}
$M \to^{k+1} M'$ then $M \to^k M_1 \to M'$ by induction hypothesis $\emptyset \vdash M_1: T$,
thus by Subject Reduction Theorem $\Gamma \vdash M': T$.\qed

\subsubsection*{Safety Theorem:}

By Corollary of the Subject Reduction Theorem $\emptyset \vdash M': T$, by the Progress Theorem
either $M'$ value or $\exists\, M".\, M' \to M"$ but since by hypothesis $M' \not\to$ then
$M'$ value.\qed

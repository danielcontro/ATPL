\subsection{}

If instead of the rule (T-APP) the following rule was defined, would the safety
theorem still be true?
\[
	\infer[\text{\small{(APP')}}]
	{\Gamma \vdash \APP{M}{N}: T}
	{
		\Gamma \vdash M: T \to T &
		\Gamma \vdash N: T
	}
\]

The Safety Theorem states that if $\emptyset \vdash M: T$ and $M \to^\ast M'.\, M' \not\to$ then
$M'$  is a value.
In order to prove it we have to prove the Progress Theorem and the the corollary of the
Subject Reduction Theorem with the updated typing rules.

\subsubsection*{Progress Theorem:}

If $\emptyset \vdash M: T$ then $M$ is a value or $\exists\, M'.\, M \to M'$
By induction on the height of the derivation tree of $\emptyset \vdash M: T$:

\paragraph*{Base cases:\\}
Analogous to the original proof

\paragraph*{Inductive cases $h = h_0 + 1$:\\}

The inductive hypothesis is that for all the derivations of height $h \le h_0$ if
$\emptyset \vdash M: T$ then $M$ is a value or $\exists\, M'.\, M \to M'$. The only case which
differs is if the last rule in the derivation is (T-APP') with $M = \APP{M_1}{M_2}$ thus the
conclusion of the rule is $\emptyset \vdash \APP{M_1}{M_2}: T$, and the premises are
$\emptyset \vdash M_1 : T \to T$ and $\emptyset \vdash M_2: T$ both of height at most $h_0$.
The inductive hypothesis can be applied to the first judgement thus:
\begin{itemize}
	\item $M_1$ value: by the lemma of Canonical Forms $M_1 = \FN{x:T}{M_{1_1}}$ hence
	      $M = \APP{\FN{x:T}{M_{1_1}}}{M_2}$. By applying the inductive hypothesis to the second
	      judgement either $M_2$ is a value $v_2$ hence the (BETA) rule can be applied obtaining
	      $M \to M_{1_1}\SUBST{x}{v_2}$ or (APP2) applies obtaining
	      $M \to \APP{\FN{x:T}{M_{1_1}}}{M_2'}$
	\item $M_1 \to M1'$: (APP1) applies, thus $M \to \APP{M_1'}{M_2}$
\end{itemize}

\subsubsection*{Substitution Lemma:}

If $\Gamma, x: S \vdash M: T$ and $\Gamma \vdash N: S$ then $\Gamma \vdash M\SUBST{x}{N}: T$
By induction on the height of the derivation of $\Gamma, x: S \vdash M: T$:

\paragraph*{Base cases:\\}
Analogous to the original proof

\paragraph*{Inductive cases $h = h_0 + 1$:\\}

The inductive hypothesis is that for all the derivations of height $h \le h_0$ if
$\Gamma, x: S \vdash M: T$ and $\Gamma \vdash N: S$ then $\Gamma \vdash M\SUBST{x}{N}: T$
is derivable.
The only case which differs is if the last rule in the derivation is (T-APP') with
$M = \APP{M_1}{M_2}$ thus the conclusion of the rule is $\Gamma, x: S \vdash \APP{M_1}{M_2}: T$,
and the premises are:
\begin{itemize}
	\item $\Gamma, x: S \vdash M_1 : T \to T$
	\item $\Gamma, x: S \vdash M_2: T$
\end{itemize}

The inductive hypothesis can be applied to both the premises obtaining
\begin{itemize}
	\item $\Gamma \vdash M_1\SUBST{x}{N}: T \to T$
	\item $\Gamma \vdash M_2\SUBST{x}{N}: T$
\end{itemize}
hence the (T-APP') rule can be applied obtaining
$\Gamma \vdash \APP{M_1\SUBST{x}{N}}{M_2\SUBST{x}{N}}: T$\qed

\subsubsection*{Inversion Lemma}

In order to properly prove this statement we have to firstly change a
bit the inversion lemma about the (T-APP) rule,
\begin{itemize}
\item[\textbf{from}] If \(\Gamma \vdash M N \TYPE{T}\) is derivable,
  then \(\exists T_1\) s.t. \(\Gamma \vdash M \TYPE{T_1 \to T}\) and
  \(\Gamma \vdash N \TYPE{T_1}\) are derivable
\item[\textbf{to}] If \(\Gamma \vdash M N \TYPE{T}\) is derivable,
  then \(\Gamma \vdash M \TYPE{T \to T}\) and \(\Gamma \vdash N
  \TYPE{T}\) are derivable
\end{itemize}

\subsubsection*{Subject Reduction Theorem:}

If $\Gamma \vdash M: T$ and $M \to M'$ then $\Gamma \vdash M': T$.
By induction on the height of the derivation of $M \to M'$:

\paragraph*{Base cases:\\}

Analogous to the original proof

\paragraph*{Inductive cases:\\}
We have to edit the 2 cases in which in the orginal proof we needed
the (T-APP) rule, and show that the theorem still works even when
using the (APP') rule. The 2 cases are (APP 1) and (APP 2) were the
lastly appllied rule:
\begin{itemize}
\item[(APP 1)] was the lastly applied rules in the derivation: the
  conclusion must be the judgment \(\APP{M_1}{M_2} \to
  \APP{M_1'}{M_2}\) and the premise \(M_1 \to M_1'\) has derivation of
  height \(k\). From the hypothesis \(\Gamma \vdash M \TYPE{T}\), by
  inversion lemma we have that \(\Gamma \vdash M_1 \TYPE{T \to T}\)
  and \(\Gamma \vdash M_2 \TYPE{T}\). From \(\Gamma \vdash M_1 \TYPE{T
    \to T}\) and \(M_1 \to M_1'\), which has a derivation height of
  \(k\) and the inductive hypothesis we can conclude that \(\Gamma
  \vdash M_1' \TYPE{T \to T}\). We can therefore apply (T-APP) rule
  and say \(\Gamma \vdash \APP{M_1'}{M_2} \TYPE{T}\), which is our
  thesis
\item[(APP 2)] This case is again similar to the previous one
\end{itemize}


\subsubsection*{Corollary of Subject Reduction Theorem:}

If $\emptyset \vdash M: T$ and $M \to M'$ then $\emptyset \vdash M': T$
By induction on the number of the reduction steps $k$ in $M \to M'$.

\paragraph*{$k = 0$:} $M' = M$ thus immediate
\paragraph*{$k = k_0 + 1$:\\}
$M \to^{k+1} M'$ then $M \to^k M_1 \to M'$ by induction hypothesis $\emptyset \vdash M_1: T$,
thus by Subject Reduction Theorem $\Gamma \vdash M': T$.\qed

\subsubsection*{Safety Theorem:}

By Corollary of the Subject Reduction Theorem $\emptyset \vdash M': T$, by the Progress Theorem
either $M'$ value or $\exists\, M".\, M' \to M"$ but since by hypothesis $M' \not\to$ then
$M'$ value.\qed

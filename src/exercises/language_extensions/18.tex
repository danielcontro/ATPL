\subsection{}

Study a variant of the language that admits a default case in the
pattern matching operation.

Evaluation would be simply to add, only one rule would be needed:
\[
\infer[j\not\in1\dots n]
      {
        \MATCH{\VAR{\ell_j = v_j}}{\text{case } \ell_i = x_i \Rightarrow M_i ^{i\in 1 \dots n}, \text{default } x_{n+1} \Rightarrow M_{n+1}}
        \to
        M_{n+1}\SUBST{x_{n+1}}{v_j}
      }
      {\text{\small{(MATCH-DEFAULT)}}}
\]

For typing on the other hand, things are a little bit more
complicated. One solution is to constrain the non-explicit cases to
fall into the same type, otherwise we would have no guarantees for
\(M_{n+1}\) to type:

(TYPE-MATCH-2)
\[
\infer
    {\Gamma \vdash
      \MATCH{M}{\text{case } \ell_i = x_i \Rightarrow M_i,
        \text{default } x_j \Rightarrow M_j
        \quad ^{i\in I\quad j\not\in I}}\TYPE{T}}
    {
      \Gamma \vdash M \TYPE{\VAR{\ell_i \TYPE{T_i} ^{i\in I}, \ell_j : T_s ^{j\in J}}}
      &
      \forall i \in I \; \Gamma, x_i \TYPE{T_i} \vdash M_i \TYPE{T}
      &
      \Gamma, x_j \TYPE{T_s} \vdash M_j \TYPE{T}
    }
\]

The rule says that every possibility of the variant type not
explicitly stated in the cases, but left to the default case, must
match the same type \(T_s\) in such a way that the statement \(M_j\)
with free variable \(x_j\) of type \(T_s\) matches the type \(T\).

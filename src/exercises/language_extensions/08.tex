\subsection{}

Prove the safety theorem for the language containing integers,
Booleans, functions and records.\\~\\ In order to prove the Safety
Theorem the Progress Theorem, Substitution Lemma and Subject Reduction
Theorem has to be proved.

\paragraph*{Progress Theorem:\\}
The proof will proceed by induction on the height of the derivation of
\(\Gamma \vdash M: T\) The base cases are left unchanged, only the
inductive ones has to be extended with the newly added typing rules:

\begin{itemize}
\item (TYPE RECORD): \(M= \REC{l_i=M_i^{i \in 1..n}}\), \(\emptyset
  \vdash M: T\) then by Inv. Lemma \(\forall\, i \in 1..n.\, \emptyset
  \vdash M_i: T_i\) with height \(\le k_0\), thus for all \(M_i\) the
  inductive hypothe can be applied. Two cases has to be distinguished:
  \begin{itemize}
  \item \(\forall\, i \in 1..n.\, M_i = v_i\): hence \(M =
    \REC{l_i=v_i^{i \in 1..n}}\) thus \(M\) is a value
  \item \(\exists\, j \in 1..n.\, M_j \to M_j'\): thus the (EVAL
    RECORD) rule can be applied obtaining \(M'\)
  \end{itemize}
\item (TYPE SELECT): \(M = \SEL{M_1}{l_j}: T_j\), \(\emptyset \vdash
  M: T\) then by Inv. Lemma\\ \(\exists\, n.\, j \in 1..n \AND
  \exists\, l_1,..,l_n\ T_1,..,T_n.\, \emptyset \vdash M_1 :
  \TREC{l_i: T_i^{i \in 1..n}}\), since the derivation of the type of
  \(M_1\) has height \(\le k_0\) the inductive hypothesis can be
  applied, hence two cases has to be distinguished:
  \begin{itemize}
  \item \(M_1\) is a value, hence \(M_1 = \REC{l_i = v_i^{i \in
      1..n}}\) thus the (SELECT) axiom can be applied with
    \(\SEL{M_1}{l_j} \to v_j = M'\)
  \item \(\exists\, M_1'.\, M_1 \to M_1'\), hence the (EVAL SELECT)
    rule can be applied obtaining\\ \(\SEL{M_1}{l_j} \to
    \SEL{M_1'}{l_j} = M'\)
  \end{itemize}
\end{itemize}

\paragraph*{Substitution Lemma:\\}
If \(\Gamma, x: S \vdash M: T\) and \(\Gamma \vdash N: S\) then \(\Gamma
\vdash M\SUBST{x}{N}: T\).  The base cases are left unchanged, only the
inductive ones has to be extended with the newly added typing rules.
By induction on the height of the derivation of \(\Gamma, x: S \vdash
M: T\):
\begin{itemize}
\item (TYPE RECORD): by Inv. Lemma \(\forall\, i \in 1..n.\, \Gamma,
  x: S \vdash M_i: T_i\) with height of the derivation \(\le k_0\)
  thus the inductive hypothesis can be applied obtaining\\ \(\forall\,
  i \in 1..n.\, \Gamma \vdash M_i\SUBST{x}{N}: T_i\) hence the (TYPE
  RECORD) rule can be applied obtaining \(\Gamma \vdash
  \REC{l_i=M_i\SUBST{x}{N}^{i \in 1..n}}= M\SUBST{x}{N}: T\)
\item (TYPE SELECT): (\(\Gamma, x: S \vdash \SEL{M_1}{l_j}: T_j\)) by
  Inv. Lemma\\ \(\exists\, n.\, j \in 1..n \AND \exists\,
  l_1,..,l_n\ T_1,..,T_n.\, \Gamma, x: S \vdash M_1 : \TREC{l_i:
    T_i^{i \in 1..n}}\) with height of the derivation \(\le k_0\)
  hence the inductive hyp. can be applied obtaining \(\Gamma \vdash
  M_1\SUBST{x}{N} : \TREC{l_i: T_i^{i \in 1..n}}\) thus the (TYPE
  SELECT) rule can be used in order to get \(\Gamma \vdash
  \SEL{M_1\SUBST{x}{N}}{l_j} = M': T_j\)
\end{itemize}


\paragraph*{Subject Reduction Theorem:\\}

If \(\Gamma \vdash M: T\) and \(M \to M'\) then \(\Gamma \vdash M':
T\).  The base cases (SUM), (IF-TRUE) and (BETA) are left unchanged,
we have to add a new base case:
\begin{itemize}
\item (SELECT) was the lastly applied rule, it menas that the
  conclusion is \(\REC{\ell_i = v_i \quad^{i\in 1\dots n}}.\ell_j \to
  v_j\) by hypothesis we know \[\Gamma \vdash \REC{\ell_i = v_i
    \quad^{i\in 1\dots n}} \TYPE{\REC{\ell_i \TYPE{T_i} \quad^{i\in
        1\dots j-1}, \ell_j \TYPE{T_j}, \ell_k \TYPE{T_k} \quad^{k\in
        j+1\dots n}}}\] so by inversion lemma we have \(\forall i \in
  1 \dots n \quad \Gamma\vdash v_i\TYPE{T_i}\) which in particular
  means \(j\in 1\dots n \Rightarrow \Gamma \vdash v_j \TYPE{T_j}\)
  which is our thesis
\end{itemize}

Inductive cases, working by induction on the height of the derivation
of \(M \to M'\) also have to be updated with the new evaluation rules:
\begin{itemize}
\item (EVAL SELECT) was the lastly applied rule, therefore the
  conclusion is \[M.\ell_j \to M'.\ell_j\] and the premise is \(M\to
  M'\) which has an height of derivation at most \(k\). By considering
  the hypothesis \(\Gamma\vdash M.\ell_j \TYPE{T_j}\) we have by
  inversion lemma that \(\Gamma\vdash M \TYPE{\REC{\ell_i \TYPE{T_i}
      \quad^{i \in 1 \dots n}}}\) with \(j\in 1\dots n\). We can apply
  the inductive hypothesis and say that \(\Gamma\vdash M'
  \TYPE{\REC{\ell_i \TYPE{T_i} \quad^{i \in 1 \dots n}}}\) but this
  means by (TYPE SELECT) that \(\Gamma \vdash M'.\ell_j \TYPE{T_j}\)
  which is our thesis.
\item (EVAL RECORD) was the lastly applied rule, therefore the
  conclusion (\(M \to M'\)) is \[\REC{\ell_i = v_i \quad^{i\in 1\dots
      j-1}, \ell_j = M_j, \ell_k = M_k \quad^{k\in j+1\dots n}} \to
  \REC{ \ell_i = v_i \quad^{i\in 1\dots j-1}, \ell_j = M_j', \ell_k =
    M_k \quad^{k\in j+1\dots n}}\] and the hypothesis is that \(M_j\to
  M_j'\), derivable with a derivation tree of height at most \(k\). by
  hypothesis \(\Gamma \vdash M \TYPE{T}\), which by inversion lemma
  means \(T = \REC{\ell_i \TYPE{T_i} \quad^{i \in 1 \dots n}}\), which
  is derivable, so by inversion lemma again we have that \(\forall i
  \in 1 \dots n \quad \Gamma\vdash M_i \TYPE{T_i}\), in particular
  \(\Gamma \vdash M_j \TYPE{T_j}\). By induction hypothesis \(\Gamma
  \vdash M_j' \TYPE{T_j}\), which means \[\Gamma \vdash \REC{\ell_i =
    v_i \quad^{i\in 1 \dots j-1}, \ell_j = M_j', \ell_k = M_k
    \quad^{k\in j+1, \dots n}} \TYPE{\REC{\ell_i \TYPE{T_i}
      \quad^{i\in 1 \dots j-1}, \ell_j \TYPE{T_j}, \ell_k \TYPE{T_k}
      \quad^{k\in j+1, \dots n}}}\] which is exactly \(T\), which
  makes the thesis.
\end{itemize}

\subsection{}

Consider the following variant and rule (TYPE MATCH):
\[
\infer[\text{\small{(TYPE MATCH GEN 2)}}] { \Gamma \vdash
  \MATCH{M}{\CASE{l_i}{x_i}{M_i^{i \in I}}}: T } { \Gamma \vdash M:
  \TVAR{l_i: T_i^{i \in J}} & J \subseteq I & \forall\, i \in
  J.\,\Gamma, x_i: T_i \vdash M_i: T }
\]

Prove that the safety theorem still holds or give a
counterexample.\\~\\

\paragraph*{Progress theorem:\\}

If \(\emptyset \vdash M: T\) then either \(M\) is a value or
\(\exists\, M'.\, M \to M'\).\\ By induction on the height of the
derivation of \(\emptyset \vdash M: T\): The base cases are left
unchanged since no new typing axiom has been added, only the inductive
ones has to be updated with the new rule, based on the last rule of
the derivation applied.
\paragraph*{\textmd{(TYPE MATCH GEN 2):}}
hence \(M = \MATCH{N}{\CASE{l_i}{x_i}{M_i^{i \in I}}}\), by Inv. Lemma
\(\exists\, J, T_i^{i \in J}\):
\begin{itemize}
\item \(\emptyset \vdash N: \TVAR{l_i: T_i^{i \in J}}\) with height of
  the derivation \(\le k_0\)
\item \(J \subseteq I\)
\item \(\forall\, i \in J.\, \emptyset, x_i: T_i \vdash M_i: T\) with
  height of the derivation \(\le k_0\)
\end{itemize}
thus the inductive hyp. can be applied to \(N\):
\begin{itemize}
\item \(N\) value: by Lemma of Canonical forms \(N = \VAR{l_i =
  v_i}\), the (MATCH) rule can be applied hence \(\exists\, M'.\, M
  \to M'\)
\item \(\exists\, N'.\, N \to N'\): the (RED MATCH) rule can be
  applied hence \(\exists\, M'.\, M \to M'\)
\end{itemize}

\paragraph*{Substitution Lemma:\\}
If \(\Gamma, x: S \vdash M: T\) and \(\Gamma \vdash N: S\) then
\(\Gamma \vdash M\SUBST{x}{N}: T\).  The base cases are left
unchanged, only the inductive ones has to be extended with the newly
added typing rules.  By induction on the height of the derivation of
\(\Gamma, x: S \vdash M: T\):
\paragraph*{\textmd{(TYPE MATCH GEN 2):}}
hence \(M = \MATCH{M_a}{\CASE{l_i}{x_i}{M_i^{i \in I}}}\), by
Inv. Lemma \(\exists\, J, T_i^{i \in J}\):
\begin{itemize}
\item \(\Gamma, x: S \vdash M_a: \TVAR{l_i: T_i^{i \in J}}\) with
  height of the derivation \(\le k_0\)
\item \(J \subseteq I\)
\item \(\forall\, i \in J.\, \Gamma, x: S, x_i: T_i \vdash M_i: T\)
  with height of the derivation \(\le k_0\) (if \(\exists\, i \in
  1..n.\, x = x_i\) or \(x_i \in fv(N)\) apply \(\alpha\)-conversion)
\end{itemize}
thus the inductive hyp. can be applied to \(M_a\) and all the \(M_i\)
obtaining
\begin{itemize}
\item \(\Gamma \vdash M_a\SUBST{x}{N}: \TVAR{l_i: T_i^{i \in J}}\)
  with height of the derivation \(\le k_0\)
\item \(\forall\, i \in J.\, \Gamma, x_i: T_i \vdash M_i\SUBST{x}{N}:
  T\) with height of the derivation \(\le k_0\)
\end{itemize}
hence the (TYPE MATCH GEN 2) rule can be applied to derive \(\Gamma
\vdash \MATCH{M_a\SUBST{x}{N}}{\CASE{l_i}{x_i}{M_i\SUBST{x}{N}^{i \in
      I}}} = M\SUBST{x}{N}: T\)

\paragraph*{Type preservation:\\}
If \(\Gamma \vdash M: T\) and \(M \to M'\) then \(\Gamma \vdash M':
T\). We'll work by induction on the height of the derivation tree of
\(M \to M'\).

\paragraph*{New base cases:\\}
\begin{itemize}
\item (MATCH) was the lastly applied rule. The conclusion is
  that \[\infer[j\in 1\dots n]{\MATCH{\VAR{\ell_j = v_j} }{
      \CASE{\ell_i}{x_i}{M_i} \quad^{i\in 1\dots n}} \to
    M_j\SUBST{x_j}{v_j}}{}\] by hypothesis we know that \(\exists T
  \mid \Gamma \vdash \MATCH{\VAR{\ell_j = v_j}}{
    \CASE{\ell_i}{x_i}{M_i} \quad^{i\in 1\dots n}} \TYPE{T}\) is
  derivable. By inversion lemma \(\Gamma, x_i \TYPE{T_i} \vdash M_i
  \TYPE{T}\) is derivable. We can apply the substitution lemma and say
  that \(\forall j \in 1 \dots n \quad \Gamma \vdash M_j
  \SUBST{x_j}{N_j} \TYPE{T}\), which is our thesis.
\end{itemize}

\paragraph*{New inductive cases:\\}
\begin{itemize}
\item (RED-MATCH) was the lastly applied rule. The conclusion is
  that \[\MATCH{M}{\CASE{\ell_i}{x_i}{M_i} \quad ^{i\in 1\dots n}} \to
  \MATCH{M'}{\CASE{\ell_i}{x_i}{M_i} \quad ^{i\in 1\dots n}}\] and the
  hypothesis is that \[M\to M'\] we also know by hypothesis
  that \[\Gamma \vdash \MATCH{M}{\CASE{\ell_i}{x_i}{M_i} \quad ^{i\in
      1\dots n}} \TYPE{T}\] for some T. This means by inversion lemma
  that \(\Gamma \vdash M: \TVAR{l_i: T_i^{i \in J}}\) and \(\forall\,
  i \in J.\,\Gamma, x_i: T_i \vdash M_i: T\) are both derivable for
  some \(J\subseteq I\), with \(I\) the set of all labels in the
  original match statement. Since \(M\to M'\) with a derivation tree
  of height at most \(k\) we can apply the inductive hypothesis and
  say that \(\Gamma \vdash M' \TYPE{\TVAR{l_i: T_i^{i \in J}}}\),
  therefore by applying (TYPE MATCH GEN 2) we can say that
  \(\Gamma\vdash \MATCH{M'}{\CASE{\ell_i}{x_i}{M_i} \quad ^{i\in
      1\dots n}} \TYPE{T}\), which is our thesis.
\item (VARIANT) was the lastly applied rule. Therefore the conclusion
  is \[\VAR{\ell = M} \to \VAR{\ell = M'}\] and the hypothesis is that
  \(M\to M'\). We know by hypothesis that \(\Gamma \vdash \VAR{\ell =
    M} \TYPE{T}\) is derivable for some type T. By lemma of canonical
  forms the term has type \(T = \VAR{\ell_i \TYPE{T_i} \quad^{i\in
      1\dots n}} \mid \ell = \ell_i\) for some \(i\in 1\dots n\). By
  inversion lemma \(\Gamma \vdash M \TYPE{T_j}\). We can then apply
  the inductive hypothesis and say \(\Gamma \vdash M' \TYPE{T_j}\),
  and therefore \(\Gamma \vdash \VAR{\ell = M'}\TYPE{\VAR{\ell_i
      \TYPE{T_i} \quad^{i\in 1\dots n}}}\) which is our thesis.
\end{itemize}

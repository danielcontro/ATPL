\subsection{}
Update the proofs of the theorems of progress, type preservation and
the substitution lemma by considering the extended language with pair
terms.

\subsubsection*{Progress theorem:}

The base cases are left unchanged, only the inductive ones has to be
extended with the newly added typing rules:
\begin{itemize}
\item (T-PAIR) was the lastly applied rule: the conclusion is
  \(\emptyset \vdash (M_1, M_2) \TYPE{T_1 * T_2}\) and the premises
  are \(\emptyset \vdash M_1 \TYPE{T_1}\) and \(\emptyset \vdash M_2
  \TYPE{T_2}\), which are derivable with height at most \(k_0\). By
  induction hypothesis \(M_1\) is either a good final value or
  \(\exists M_1' \mid M_1 \to M_1'\), same for \(M_2\). 3 cases
  emerge:
  \begin{itemize}
  \item \(M_1 = v_1, M_2 = v_2\) therefore \(\PAIR{v_1}{v_2}\) is a
    good final value, the theorem holds
  \item \(M_1 = v_1, \exists M_2'\mid M_2 \to M_2'\). Therefore by
    applying (EVAL-PAIR-2) \(\PAIR{v_1}{M_2} \to \PAIR{v_1}{M_2'} =
    M'\)
  \item \(\exists M_1'\mid M_1 \to M_1'\). Therefore by applying
    (EVAL-PAIR-1) \(\PAIR{M_1}{M_2} \to \PAIR{M_1'}{M_2} = M'\)
  \end{itemize}
\item (T-PROJ-1) was the lastly applied rule, then the conclusion is
  \(\emptyset \vdash \FST{M_1} \TYPE{T_1}\) and the premise is
  \(\emptyset \vdash M_1 \TYPE{T_1 * T_2}\). by induction hypothesis
  we have 2 cases:
  \begin{itemize}
  \item \(M_1\) is a good value, then by canonical forms lemma \(M_1 =
    \PAIR{v_1}{v_2}\), therefore by applying (PAIR-1)
    \(\FST{\PAIR{v_1}{v_2}} \to v_1 = M'\)
  \item \(\exists M_1' \mid M_1 \to M_1'\), therefore by applying
    (PROJECT-1) \(\FST{M_1} \to \FST{M_1'}\)
  \end{itemize}
\item (T-PROJ-2) is analogous to (T-PROJ-1)
\end{itemize}
So in every case the statement holds.

\subsubsection*{Substitution Lemma:}

The base cases are left unchanged, only the inductive ones has to be
extended with the newly added typing rules:
\begin{itemize}
\item (T-PAIR): \(M = \PAIR{M_1}{M_2}\), if \(\Gamma, x: S \vdash M: T\)
  then by Inv. lemma' \(\Gamma \vdash M_1: T_1\) and \(\Gamma \vdash M_2:
  T_2\), which derivation trees have height \(\le k_0\) thus the
  inductive hyp. can be applied. The (T-PAIR) rule can be applied to
  the \(M_1\SUBST{x}{N}\) and \(M_2\SUBST{x}{N}\) obtaining\\ \(\Gamma
  \vdash \PAIR{M_1\SUBST{x}{N}}{M_2\SUBST{x}{N}} = M\SUBST{x}{N} : T_1
  * T_2\)
\item (T-PROJECT 1): \(M = \FST{M'}\), \(\Gamma, x:S \vdash \FST{M'} :
  T_1\) then by Inv. Lemma \(\Gamma \vdash M' : T_1 \ast T_2\) derivation
  tree has height \(\le k_0\) thus the inductive hyp. can be applied,
  hence\\ \(\Gamma \vdash M'\SUBST{x}{N}: T_1 \ast T_2\) so the
  (T-PROJECT 1) rule can be applied obtaining\\ \(\Gamma \vdash
  \FST{M'\SUBST{x}{N}} = M\SUBST{x}{N}: T_1\)
\item (T-PROJECT 2): \(M = \SND{M'}\), \(\Gamma, x:S \vdash \SND{M'} :
  T_2\) then by Inv. Lemma \(\Gamma \vdash M' : T_1 \ast T_2\) derivation
  tree has height \(\le k_0\) thus the inductive hyp. can be applied,
  hence\\ \(\Gamma \vdash M'\SUBST{x}{N}: T_1 \ast T_2\) so the
  (T-PROJECT 2) rule can be applied obtaining\\ \(\Gamma \vdash
  \SND{M'\SUBST{x}{N}} = M\SUBST{x}{N}: T_2\)
\end{itemize}

\subsubsection*{Type preservation:}
If \(\Gamma \vdash M: T\) and \(M \to M'\) then \(\Gamma \vdash M': T\). By
induction on the height of the derivation of \(M \to M'\):

\paragraph*{New base cases:\\}
\begin{itemize}
\item[(PAIR 1)] was the lastly applied rule, therefore the conclusion
  is \(\FST{\PAIR{v_1}{v_2}} \to v_1\) and there are no
  premises. \(\Gamma \vdash \FST{\PAIR{v_1}{v_2}} \TYPE{T}\) is
  derivable, then by the following tree \[\infer[\text{(T-PROJECT
      1)}]{\Gamma \vdash \FST{\PAIR{v_1}{v_2}}
    \TYPE{T}}{\infer[\text{(T-PAIR)}]{\Gamma \vdash \PAIR{v_1}{v_2}
      \TYPE{T * T_1}}{\infer{\Gamma \vdash v_1 \TYPE{T}}{\pi_1} &
      \infer{\Gamma \vdash v_2 \TYPE{T_1}}{\pi_2}}}\] \(\Gamma \vdash
  v_1 \TYPE{T}\) is derivable by some derivation tree \(\pi_1\).
\item[(PAIR 2)] This case is the same as the last one, but for the
  second element of the pair.
\end{itemize}

\paragraph*{New inductive cases:\\}
\begin{itemize}
\item[(PROJECT 1)] was the lastly applied rule, therefore the
  conclusion is \(\FST{M} \to \FST{M'}\), and it has a derivation
  height of $k$. and the hypothesis is \(M\to M'\). By hypothesis
  \(\Gamma \vdash \FST{M} \TYPE{T}\). By inversion lemma on pairs, we
  can say that \(\exists T_1 \mid \Gamma \vdash M \TYPE{T *
    T_1}\). Since \(M\to M'\) has a derivation height of \(k\), we can
  apply the inductive hypothesis and say \(\Gamma \vdash \FST{M'}
  \TYPE{T}\), which is our thesis.
\item[(PROJECT 2)] was the lastly applied hypothesis. This case is
  similar to the previous one.
\item[(EVAL PAIR 1)] was the lastly applied rule: the conclusion is
  \(\PAIR{M_1}{M_2} \to \PAIR{M_1'}{M_2}\) and the premise is that
  \(M_1 \to M_1'\) is derivable with a derivation of height
  \(k\). \(\Gamma \vdash \PAIR{M_1}{M_2} \TYPE{T}\) is derivable for
  some type \(T\). Because of lemma of canonical forms, this type is
  \(T_1 * T_2\) for some \(T_1, T_2\). This means that we can apply
  (T-PAIR) and say that \(\Gamma \vdash M_1 \TYPE{T_1}\) and \(\Gamma
  \vdash M_2 \TYPE{T_2}\). We can now apply inductive hypothesis and
  state that \(\Gamma \vdash M_1' \TYPE{T_1}\), and therefore \(\Gamma
  \vdash \PAIR{M_1'}{M_2} \TYPE{T}\), which is our thesis.
\item[(EVAL PAIR 2)] was the lastly applied rule: this case is
  analogous to the previous one.
\end{itemize}

\subsection{}

Define inversion lemmas for the extended language with variant
types. Prove the safety theorem for the new language.

\subsubsection*{Inversion lemma:}

\begin{itemize}
\item If \(\Gamma \vdash \VAR{l_j = M}: \TVAR{l_i: T_i^{i \in 1..n}}\)
  then \(\exists\, j \in 1..n.\, \Gamma \vdash M: T_j\)
\item If \(\Gamma \vdash \MATCH{M}{\CASE{l_i}{x_i}{M_i^{i \in 1..n}}}:
  T\) then \(\exists\, T_1,..,T_n.\, \Gamma \vdash M: \TVAR{l_i:
    T_i^{i \in 1..n}}\) and \(\forall\, i \in 1..n.\, \Gamma, x_i: T_i
  \vdash M_i: T\)
\end{itemize}


\subsubsection*{Progress Theorem:}

If \(\emptyset \vdash M: T\) then \(M\) is a value or \(\exists\,
M'.\, M \to M'\)\\ By induction on the height of the derivation tree
of \(\emptyset \vdash M: T\):\\ The base cases are left unchanged,
only the inductive ones has to be extended with the newly added typing
rules.
\begin{itemize}
\item (TYPE VARIANT): if \(\emptyset \vdash \VAR{l_j = M_1} = M:
  \TVAR{l_i:T_i^{i \in 1..n}}\) then by Inv. Lemma \(\exists\, j \in
  1..n.\, \emptyset \vdash M_1: T_j\) with height of the derivation
  \(\le k_0\) thus the inductive hypothesis can be applied; if \(M_1\)
  is a value then \(M = \VAR{l_j = v_j}\) is a value, otherwise the
  VARIANT rule can be applied obtaining \(M' = \VAR{l_j = M_1'}\)
\item (TYPE MATCH): if \(\emptyset \vdash
  \MATCH{M_{11}}{\CASE{l_i}{x_i}{M_i^{i \in 1..n}}}\) then by
  Inv. Lemma \(\exists\, T_1,..,T_n.\, \emptyset \vdash M_{11}:
  \TVAR{l_i: T_i^{i \in 1..n}}\) and \(\forall\, i \in 1..n.\, x_i:
  T_i \vdash M_i: T\) with height of the derivation \(\le k_0\), hence
  the inductive hyp. can be applied to \(M_{11}\) which is either a
  value \(\VAR{l_j = v_j}\) in which case the (MATCH) rule can be
  applied obtaining \(M' = M_j\SUBST{x_j}{v_j}\) or \(\exists\,
  M_{11}'.\, M_{11} \to M_{11}'\) hence the (RED MATCH) rule can be
  applied obtaining \(\MATCH{M_{11}'}{\CASE{l_i}{x_i}{M_i^{i \in
        1..n}}} = M'\)
\end{itemize}


\subsubsection*{Substitution Lemma:}

If \(\Gamma, x: S \vdash M: T\) and \(\Gamma \vdash N: S\) then
\(\Gamma \vdash M\SUBST{x}{N}: T\)\\ By induction on the height of the
derivation of \(\Gamma, x: S \vdash M: T\):\\ The base cases are left
unchanged, only the inductive ones has to be extended with the newly
added typing rules.
\begin{itemize}
\item (TYPE VARIANT): \(\Gamma, x: S \vdash \VAR{l_j = M_1}:
  \TVAR{l_i: T_i^{i \in 1..n}}\), by Inv. Lemma \(\exists\, j \in
  1..n.\, \Gamma, x: S \vdash M_1: T_j\) with height of the derivation
  \(\le k_0\) hence the inductive hyp. can be applied obtaining
  \(\Gamma \vdash M_1\SUBST{x}{N}: T_j\) and now the (TYPE VARIANT)
  rule can be applied getting \(\Gamma \vdash \VAR{l_j =
    M_1\SUBST{x}{N}} = M\SUBST{x}{N} : T\)
\item (TYPE MATCH): \(\Gamma, x: S \vdash
  \MATCH{M'}{\CASE{l_i}{x_i}{M_i^{i \in 1..n}}}: T\), by Inv. Lemma
  \(\exists\, T_1,..,T_n.\, \Gamma, x: S \vdash M': \TVAR{l_i: T_i^{i
      \in 1..n}}\) and \(\forall\, i \in 1..n.\, \Gamma, x: S, x_i:
  T_i \vdash M_i: T\) with height of the derivation \(\le k_0\) hence
  the inductive hyp. can be applied to the premises and then the (TYPE
  MATCH) rule can be applied obtaining \(\Gamma, x: S \vdash
  \MATCH{M'\SUBST{x}{N}}{\CASE{l_i}{x_i}{M_i\SUBST{x}{N}^{i \in
        1..n}}} = M\SUBST{x}{N}: T\)
\end{itemize}

\subsubsection*{Inversion Lemma}
Since new typing rules were added, new inversion lemma statements can
be proven:
\begin{itemize}
\item if \(\Gamma\vdash \MATCH{M}{\CASE{\ell_i}{x_i}{M_i}\quad^{i\in
    1\dots n}}\TYPE{T}\) is derivable then
  \begin{itemize}
  \item \(\Gamma\vdash M \TYPE{\VAR{\ell_i \TYPE{T_i} \quad^{i\in 1
        \dots n}}}\)
  \item \(\Gamma, x_i\TYPE{T_i} \vdash M_i \TYPE{M_i}\quad \forall
    i\in 1\dots n\)
  \end{itemize}
  Proof: follows from the application of the (TYPE MATCH) rule
\item if \(\Gamma \vdash \VAR{\ell_j = M} \TYPE{\VAR{\ell_i \TYPE{T_i}
    \quad^{i\in 1\dots n}}}\) and \(j\in 1\dots n\) then
  \(\Gamma\vdash M\TYPE{T_j}\). Proof: it follows from the (TYPE
  VARIANT) typing rule
\end{itemize}

\subsubsection*{Subject Reduction Theorem:}

If \(\Gamma \vdash M: T\) and \(M \to M'\) then \(\Gamma \vdash M':
T\). The proof will work as before by induction on the height of the
derivatipon of \(M \to M'\), we'll have to add some base cases and
inductive cases.
\paragraph*{New base cases:\\}
\begin{itemize}
\item[(MATCH)] was the lastly applied rule. The conclusion is
  that \[\infer[j\in 1\dots n]{\MATCH{\VAR{\ell_j = v_j} }{
      \CASE{\ell_i}{x_i}{M_i} \quad^{i\in 1\dots n}} \to
    M_j\SUBST{x_j}{v_j}}{}\] by hypothesis we know that \(\exists T
  \mid \Gamma \vdash \MATCH{\VAR{\ell_j = v_j}}{
    \CASE{\ell_i}{x_i}{M_i} \quad^{i\in 1\dots n}} \TYPE{T}\) is
  derivable. By inversion lemma \(\Gamma, x_i \TYPE{T_i} \vdash M_i
  \TYPE{T}\) is derivable. We can apply the substitution lemma and say
  that \(\forall j \in 1 \dots n \quad \Gamma \vdash M_j
  \SUBST{x_j}{N_j} \TYPE{T}\), which is our thesis.
\end{itemize}

\paragraph*{New inductive cases:\\}
\begin{itemize}
\item[(RED-MATCH)] was the lastly applied rule. The conclusion is that
  \(\MATCH{M}{\CASE{\ell_i}{x_i}{M_i} \quad ^{i\in 1\dots n}} \to
  \MATCH{M'}{\CASE{\ell_i}{x_i}{M_i} \quad ^{i\in 1\dots n}}\) and the
  hypothesis is that \(M\to M'\) we also know by hypothesis that
  \(\Gamma \vdash \MATCH{M}{\CASE{\ell_i}{x_i}{M_i} \quad ^{i\in
      1\dots n}} \TYPE{T}\) for some T. This means by inversion lemma
  that \(\Gamma \vdash M \TYPE{\VAR{\ell_i \TYPE{T_i}^{i\in 1\dots
        n}}}\) and \(\forall i \in 1 \dots n \quad \Gamma,
  x_i\TYPE{T_i} \vdash M_i \TYPE{T_i}\) are both derivable. In
  particular, since also \(M \to M'\) holds, we can apply the
  inductive hypothesis and state that \(\Gamma \vdash M'
  \TYPE{\VAR{\ell_i \TYPE{T_i}^{i\in 1\dots n}}}\). We can now apply
  the (TYPE MATCH) rule and observe that \(\Gamma \vdash
  \MATCH{M'}{\CASE{\ell_i}{x_i}{M_i} \quad ^{i\in 1\dots n}}
  \TYPE{T}\), which is our thesis.
\item[(VARIANT)] was the lastly applied rule. Therefore the conclusion
  is \(\VAR{\ell = M} \to \VAR{\ell = M'}\) and the hypothesis is that
  \(M\to M'\). We know by hypothesis that \(\Gamma \vdash \VAR{\ell =
    M} \TYPE{T}\) is derivable for some type T. By lemma of canonical
  forms the term has type \(T = \VAR{\ell_i \TYPE{T_i} \quad^{i\in
      1\dots n}} \mid \ell = \ell_i\) for some \(i\in 1\dots n\). By
  inversion lemma \(\Gamma \vdash M \TYPE{T_j}\). We can then apply
  the inductive hypothesis and say \(\Gamma \vdash M' \TYPE{T_j}\),
  and therefore \(\Gamma \vdash \VAR{\ell = M'}\TYPE{\VAR{\ell_i
      \TYPE{T_i} \quad^{i\in 1\dots n}}}\) which is our thesis.
\end{itemize} \qed

\subsubsection*{Safety Theorem:}

By Corollary of the Subject Reduction Theorem \(\emptyset \vdash M':
T\), by the Progress Theorem either \(M'\) value or \(\exists\, M".\,
M' \to M"\) but since by hypothesis \(M' \not\to\) then \(M'\)
value.\qed

\subsection{}

Prove the theorems of type preservation, progress and safety for the
language with exceptions.

\subsubsection*{Progress Theorem}

If \(\emptyset \vdash M: T\) then either \(M\) is a value, or
\(\exists\, M'.\, M \to M'\) or \(\exists\, v.\, M = \THROW{v}\) By
induction on the height of the derivation tree of \(\emptyset \vdash
M: T\):\\ The base cases are left unchanged, only the inductive ones
has to be extended with the derivation trees that have the newly added
typing rules as first ones.
\begin{itemize}
\item (T-RAISE): then \(M = \THROW{N}\), by Inv. Lemma
  \(\exists\, T_{exn}.\, \emptyset \vdash N: T_{exn}\) with
  height of the derivation \(\le k_0\) thus the inductive
  hyp. can be applied obtaining that either:
  \begin{itemize}
  \item \(N \to N'\): hence by (RAISE 1)
    \(\exists\, M'.\, M \to M'\)
  \item \(N\) is a value \(v\): hence \(M =
    \THROW{v}\)
  \item \(\exists\, v.\, N = \THROW{v}\): hence by
    (RAISE 2) \(\exists\, M'.\, M \to M'\)
  \end{itemize}
\item (T-TRY): then \(M = \TRY{M_1}{M_2}\), by Inv. Lemma
  \begin{itemize}
  \item \(\emptyset \vdash M_1: T\) with height of
    the derivation \(\le k_0\)
  \item \(\exists\, T_{exn}.\, \emptyset \vdash
    M_2: T_{exn} \to T\) with height of the
    derivation \(\le k_0\)
  \end{itemize}
  hence the inductive hyp. can be applied to \(M_1\)
  \begin{itemize}
  \item \(M_1 \to M_1'\): hence by (TRY)
    \(\exists\, M'.\, M \to M'\)
  \item \(M_1\) is a value \(v\): hence by (TRY
    VAL) \(\exists\, M'.\, M \to M'\)
  \item \(\exists\, v.\, M_1 = \THROW{v}\): hence
    by (TRY HANDLE) \(\exists\, M'.\, M \to M'\)
  \end{itemize}
\end{itemize}

\subsubsection*{Substitution Lemma:}

If \(\Gamma, x: S \vdash M: T\) and \(\Gamma \vdash N: S\) then
\(\Gamma \vdash M\SUBST{x}{N}: T\)\\ By induction on the height of the
derivation of \(\Gamma, x: S \vdash M: T\):\\ The base cases are left
unchanged, only the inductive ones has to be extended with the newly
added typing rules.
\begin{itemize}
\item (T-RAISE): then \(M = \THROW{M_1}\), by Inv. Lemma \(\exists\,
  T_{exn}.\, \Gamma, x: S \vdash M_1: T_{exn}\) with height of the
  derivation \(\le k_0\), hence the inductive hyp. can be applied
  obtaining \(\Gamma \vdash M_1\SUBST{x}{N}: T_{exn}\), and by
  applying the (T-RAISE) rule it can be derived \(\Gamma \vdash
  \THROW{M_1\SUBST{x}{N}} = M\SUBST{x}{N}: T\)
\item (T-TRY): then \(M = \TRY{M_1}{M_2}\) and by Inv. Lemma
  \begin{itemize}
  \item \(\Gamma, x: S \vdash M_1: T\) with height of the derivation
    \(\le k_0\)
  \item \(\exists\, T_{exn}.\, \Gamma, x: S \vdash M_2: T_{exn} \to
    T\) with height of the derivation \(\le k_0\)
  \end{itemize}
  hence the inductive hyp. can be applied to \(M_1\) and
  \(M_2\) obtaining \(\Gamma \vdash M_1\SUBST{x}{N}: T\)
  and \(\Gamma \vdash M_2\SUBST{x}{N}: T_{exn} \to T\),
  thus by the (T-TRY) rule \(\Gamma \vdash
  \TRY{M_1\SUBST{x}{N}}{M_2\SUBST{x}{N}} = M\SUBST{x}{N}:
  T\)
\end{itemize}

\subsubsection*{Type Preservation:}

If \(\Gamma \vdash M: T\) and \(M \to M'\) then \(\Gamma \vdash M':
T\). We'll work by induction on the evaluation rule \(M \to M'\). We
need to add some base cases and some inductive cases.

\paragraph*{New base cases:\\}
\begin{itemize}
\item[(TRY HANDLE)] was the lasyly applied rule, therefore \(M =
  \TRY{\THROW{v}}{M_1}\), \(M' = \APP{M_1}{v}\). By hypothesis
  \(\Gamma \vdash M \TYPE{T}\), so by inversion lemma \(\exists
  T_{exn} \mid \Gamma \vdash v \TYPE{T_{exn}}\) and \(\Gamma \vdash
  M_1 \TYPE{T_{exn} \to T}\). Therefore \(\Gamma \vdash \APP{M_1}{v}
  \TYPE{T}\), which is our thesis.
\item[(TRY VAL)] was the lastly applied rule, therefore the conclusion
  is that \(\TRY{v}{M} \to v\) and there are no hypothesis. By
  hypothesis of the theorem \(\Gamma\vdash \TRY{v}{M}\TYPE{T}\), which
  means (by inversion lemma) \(\Gamma\vdash v \TYPE{T}\), which is
  outr thesis.
\item[(RAISE 2)] was the lastly applied rule, therefore the conclusion
  is that \(\THROW{(\THROW{v})} \to \THROW{v}\). The hypothesis of the
  theorem is that \(\Gamma\vdash \THROW{(\THROW{v})} \TYPE{T}\) is
  derivable, therefore by inversion lemma \(\Gamma\vdash \THROW{v}
  \TYPE{T}\) is derivable for some tree \(\pi_1\), which is our
  thesis.
\item[(RAISE APP 1)] is the lastly applied rule, therefore the
  conlcusion is that \(\APP{(\THROW{v})}{M} \to \THROW{v}\) and there
  are no hypothesis. Since \(\Gamma\vdash \APP{(\THROW{v})}{M}
  \TYPE{T}\) is derivable by hypothesis, but by (T RAISE)
  \(\THROW{v}\) can match any type \(T\), therefore \(\Gamma \vdash
  \THROW{v} \TYPE{T}\) is derivable, which is our thesis.
\item[(RAISE APP 2)] is the lastly applied rule, the reasoning is
  analogous to the previous case.
\item[(RAISE IFTHENELSE)] is the lastly applied rule, the reasoning is
  the same as before.
\item[(RAISE SUM 1)] was the lastly applied rule, therefore the
  conclusion is \(\THROW{v} + M \to \THROW{v}\) and there are no
  hypothesis. Since \(\Gamma \vdash \THROW{v} + M \TYPE{T}\) is
  derivable for some type \(T\), we can aaply the inversion lemma and
  say that \(\Gamma \vdash \THROW{v}\TYPE{Nat}\) is derivable, which
  is our thesis.
\item[(RAISE SUM 1)] is analogous to the previous case.
\end{itemize}

\paragraph*{New inductive cases:\\}
\begin{itemize}
\item[(RAISE 1)] was the lastly applied rule, therefore the conclusion
  is that \(\THROW{M} \to \THROW{M'}\) and the hypothesis is that
  \(M\to M'\). By hypothesis also \(\Gamma\vdash \THROW{M} \TYPE{T}\)
  holds. By inversion lemma we can say that \(\exists T_{exn} \mid
  \Gamma \vdash M \TYPE{T_{exn}}\), and therefore by inductive hp
  \(\Gamma \vdash M' \TYPE{T_{exn}}\), which means that \(\exists
  T_{exn} \mid \Gamma \vdash M' \TYPE{T_{exn}}\) and therefore we can
  apply (T-RAISE) and say that \(\Gamma \vdash \THROW{M'} \TYPE{T}\),
  which is our thesis.
\item[(TRY)] was the lastly applied rule, therefore the conclusion is
  that \(\TRY{M}{N} \to \TRY{M'}{N}\) and the hypothesis is that \(M
  \to M'\) is derivable with a derivation height of at most
  \(k\). Since by hypotesis \(\Gamma \vdash \TRY{M}{N}\TYPE{T}\) by
  inversion lemma we can say that both \(\Gamma\vdash M \TYPE{T}\) and
  \(\Gamma\vdash N \TYPE{T_{exn} \to T}\) are derivable. In particular
  since \(\Gamma\vdash M \TYPE{T}\) is derivable and \(M \to M'\) is
  derivable with height at most \(k\) we can use the inductive
  hypotesis and say \(\Gamma\vdash M' \TYPE{T}\). Beacuse of that and
  the fact that \(\Gamma\vdash N \TYPE{T_{exn} \to T}\) we can apply
  (T TRY) and state that \(\Gamma\vdash \TRY{M'}{N} \TYPE{T}\), which
  is our thesis.
\end{itemize} \qed

\subsubsection*{Safety Theorem:}

By Corollary of the Subject Reduction Theorem \(\emptyset \vdash M':
T\), by the Progress Theorem either \(M'\) value or \(\exists\, M".\,
M' \to M"\) but since by hypothesis \(M' \not\to\) then \(M'\)
value.\qed
